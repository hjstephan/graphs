\documentclass[a4paper,12pt]{article}
\usepackage[utf8]{inputenc}
\usepackage[german]{babel}
\usepackage{amsmath}
\usepackage{amssymb}
\usepackage{amsthm}
\usepackage{geometry}
\usepackage{graphicx}
\usepackage[hidelinks]{hyperref}
\usepackage{listings}
\usepackage{xcolor}
\usepackage{algorithm}
\usepackage{algpseudocode}
\usepackage{chngcntr}

\geometry{margin=2.5cm}

\theoremstyle{definition}
\newtheorem{definition}{Definition}
\newtheorem{beispiel}{Beispiel}
\newtheorem{satz}{Satz}
\newtheorem{lemma}{Lemma}
\newtheorem{korollar}{Korollar}
\counterwithin{algorithm}{section}

% Code-Highlighting
\lstset{
	basicstyle=\ttfamily\small,
	keywordstyle=\color{blue},
	commentstyle=\color{gray},
	stringstyle=\color{red},
	numbers=left,
	numberstyle=\tiny,
	breaklines=true,
	frame=single
}

\title{Graphen}
\author{Stephan Epp}
\date{\today}

\begin{document}
	
\maketitle

\tableofcontents
\newpage

\section{Einführung}

Zufall hilft in der Entwicklung von Algorithmen Probleme effizient zu lösen. Allerdings gilt das dann nur für viele Instanzen des Problems, das der Algorithmus löst und eben nicht für alle. Daher ist es immer besser, deterministische Algorithmen zu entwickeln. Denn ihre Laufzeit halten sie für alle Instanzen stets ein.

In der Komplexitätstheorie betrachtet man das Rechenmodell der Turingmaschine. Eine nichtdeterministische Turingmaschine ist eine Turingmaschine, für die es eine Eingabe gibt, bei der diese Turingmaschine sich aussucht, welches der nächste Zustand ist. Das kann sie nur deshalb, weil sie so definiert ist, dass es für diese Eingabe mehr als nur einen anderen nächsten Zustand gibt. Diese Definition befreit den Anwender dieser Maschine nicht von der Unsicherheit, was diese Maschine schließlich in einer Abarbeitungsfolge berechnet. Es ist nicht erklärbar, wie dann davon auszugehen ist, dass diese Maschine nur in einer Abarbeitungsfolge richtig rechnet.



\section{Definitionen}

\subsection{Profil}
\begin{definition}[Profil]
	Das Profil eines Graphen $G=(V, E)$ besteht aus 
	\begin{itemize}
		\item[-] dem kürzesten Weg für alle Knoten $v \in V$,
		\item[-] dem längsten Weg für alle Knoten $v \in V$ und
		\item[-] dem Maß für die Anzahl der Kanten $\frac{\left|V\right|}{\left|E\right|}$.
	\end{itemize}
\end{definition}
Mit diesem Profil wird eine Verteilung definiert, in die sich alle Graphen einordnen lassen. Ausschlaggebend für die unterschiedlichen Bereiche in der Verteilung ist die erforderliche Laufzeit der Algorithmen zur Ermittlung aller Profilwerte. Natürlich ist es dafür nötig, dass die Algorithmen die Profilwerte in optimaler Laufzeit berechnen. Sonst wäre die Einordnung nicht klar.

\section{Algorithmen}

\section{Experimente}

\section{Zusammenfassung}

\end{document}